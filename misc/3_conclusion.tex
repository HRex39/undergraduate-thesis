%%
% The BIThesis Template for Bachelor Graduation Thesis
%
% 北京理工大学毕业设计(论文)结论 —— 使用 XeLaTeX 编译
%
% Copyright 2020-2021 BITNP
%
% This work may be distributed and/or modified under the
% conditions of the LaTeX Project Public License, either version 1.3
% of this license or (at your option) any later version.
% The latest version of this license is in
%   http://www.latex-project.org/lppl.txt
% and version 1.3 or later is part of all distributions of LaTeX
% version 2005/12/01 or later.
%
% This work has the LPPL maintenance status `maintained'.
%
% The Current Maintainer of this work is Feng Kaiyu.
%
% Compile with: xelatex -> biber -> xelatex -> xelatex

\unnumchapter{结~~~~论}
\renewcommand{\thechapter}{结论}

\ctexset{
  section/number = \arabic{section}
}

% 结论部分尽量不使用 \subsection 二级标题,只使用 \section 一级标题

本文提出的方法实现了基于DQN及其改进算法的自动驾驶决策器和控制器,并就DQN及其改进算法开展了较为深入的研究。DQN 算法作为无模型的强化学习算法,通过离散化动作空间,能够应用于连续状态空间。Dueling DQN因其网络结构设计较为简单,在复杂性较低的自动驾驶决策仿真环境中表现较好,Double DQN算法由于其改进的动作选择和评估方法,能够获得更加稳定有效的行为策略。

实验结果表明,通过调整DQN网络的超参数和网络结构,本文设计实现的基于DQN及其改进算法的自动驾驶决策器输出离散的决策信号,经过底层控制器的控制输出,能够完成无碰撞的自动驾驶车辆换道任务;基于DQN及其改进算法的自动驾驶控制器输出离散的控制信号,能够完成自动驾驶车辆的无碰撞循迹任务。
综上所述,本文设计实现的自动驾驶决策器和控制器达到了预期的决策和控制效果,能够有效提高自动驾驶车辆在决策和控制中的准确性和鲁棒性。本文对于DQN算法及其多种改进算法在自动驾驶车辆决策控制方面的应用,不仅为对自动驾驶领域类似的复杂动力学模型问题的解决提供了新的思路,也为自动驾驶的决策与控制提供了借鉴和参考。

% 骂人,狠狠的骂!!
但是,本文对强化学习算法在自动驾驶的决策和控制方面的研究还处在初级阶段,目前研究的是自动驾驶决策器和控制器在连续状态空间和离散动作空间的情况,对于连续状态空间和连续动作空间(如DDPG、PPO等算法)还需要进行更加深入的研究。

对于复杂多样的的传感器输入,仅使用传统DQN直接由输入灰度图像输出动作,不添加车辆信息与导航信息,控制效果不佳。未来考虑采用卷积神经网络(如ResNet\cite{targ2016resnet})对传感器信息特征值进行提取,同时,由于传感器输入需要与车辆信息和导航信息组合,所以仍需要对DQN网络结构进行更加深入的研究和改动。
