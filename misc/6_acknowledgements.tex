%%
% The BIThesis Template for Bachelor Graduation Thesis
%
% 北京理工大学毕业设计(论文)致谢 —— 使用 XeLaTeX 编译
%
% Copyright 2020-2021 BITNP
%
% This work may be distributed and/or modified under the
% conditions of the LaTeX Project Public License, either version 1.3
% of this license or (at your option) any later version.
% The latest version of this license is in
%   http://www.latex-project.org/lppl.txt
% and version 1.3 or later is part of all distributions of LaTeX
% version 2005/12/01 or later.
%
% This work has the LPPL maintenance status `maintained'.
%
% The Current Maintainer of this work is Feng Kaiyu.
%
% Compile with: xelatex -> biber -> xelatex -> xelatex

\unnumchapter{致~~~~谢}
\renewcommand{\thechapter}{致谢}

\ctexset{
  section/number = \arabic{section}
}

% 致谢部分尽量不使用 \subsection 二级标题,只使用 \section 一级标题

修修改改缝缝补补,从2022年5月2号创建的第一版Github-repo开始,历时近一个月,总算根据多方意见完成了论文写作。值此论文完成之际,我想向我的导师宋春雷表示诚挚的敬意和衷心的感谢。从论文选题伊始,老师委托丁子豪学长对我悉心的指导和建议,我也开始对论文写作和深度强化学习有了初步的了解和认识。是老师和学长的殷切教导,才使得盲审有了一个好的结果。

其次,我想感谢一路陪伴我的父母、家人,家人永远是我的坚强后盾,没有你们的帮助和支持,我不可能顺利的完成学业。纵使出现了一些变故,但你们的支持使我接续奋斗。

在大学四年的科创活动中,我特别感谢的是北京理工大学无人驾驶方程式赛车队(Beijing Institute of Technology
Driverless Racing Team, BITFSD)。感谢陈泰然学长、马宁博士、高欣彧、杨少坤、龚海龙、吕文成、封蕴籍、李云巍……没有车队这样的专注于科研和实践相结合的团队,我不会对自动驾驶产生这样的兴趣,也不会随队捧杯。

特别关键的,我想感谢北京理工大学交响乐团合唱团艺术总监贺春华教授,感谢音乐的力量。在大学四年的学习生活中,虽然我作为仅仅来唱唱歌的同学,也许只是过客,但我仍然获得了合唱艺术、声乐艺术美的感受。同时感谢合唱团的朋友们,李瀚宇、葛易谙、张柏灵、袁海洋、杨峰、王首智……“他们应该是”一群热爱音乐的朋友们,来日方长,后会有期(呵呵)。

大学四年,即将毕业,即将工作。感谢上汽通用泛亚汽车技术中心有限公司的Offer、感谢金龙联合汽车工业(苏州)有限公司提供的实习机会。

请你别忘了我,Non ti scordar di me。