%%
% The BIThesis Template for Bachelor Graduation Thesis
%
% 北京理工大学毕业设计(论文)第二章节 —— 使用 XeLaTeX 编译
%
% Copyright 2020-2021 BITNP
%
% This work may be distributed and/or modified under the
% conditions of the LaTeX Project Public License, either version 1.3
% of this license or (at your option) any later version.
% The latest version of this license is in
%   http://www.latex-project.org/lppl.txt
% and version 1.3 or later is part of all distributions of LaTeX
% version 2005/12/01 or later.
%
% This work has the LPPL maintenance status `maintained'.
%
% The Current Maintainer of this work is Feng Kaiyu.
%%

\chapter{强化学习相关算法理论分析}

\section{强化学习算法}

强化学习(Reinforcement Learning, RL)通过与环境交互,学习状态到行为的映射关系。如图\ref{强化学习原理}所示,在一个离散时间序列$t=0,1,2,… $中,智能体需要完成某项任务。在每一个时间$t$,智能体都能从环境中接受一个状态$S_t$,并通过动作$a_t$与环境继续交互,环境会产生新的状态$S_{t+1}$,同时给出一个立即回报$r_{t+1}$。如此循环下去,智能体与环境不断地交互,从而产生更多数据(状态和回报),并利用新的数据进一步改善自身的行为。


\begin{figure}[htbp]
  \vspace{13pt} % 调整图片与上文的垂直距离
  \centering
  \includegraphics[width=0.63\textwidth]{images/chapter2/RL_struction.png}
  \caption{强化学习原理}\label{强化学习原理} % label 用来在文中索引
\end{figure}

\section{Q-learning算法}

\section{DQN算法}


\section{改进的DQN算法}

\section{本章小结}


% \section{代码片段}

% \begin{lstlisting}[language=Python, caption={Python Code}, label={lst:pythonfile}]
% import numpy as np

% def incmatrix(genl1,genl2):
%     m = len(genl1)
%     n = len(genl2)
%     M = None #to become the incidence matrix
%     VT = np.zeros((n*m,1), int)  #dummy variable

%     #compute the bitwise xor matrix
%     M1 = bitxormatrix(genl1)
%     M2 = np.triu(bitxormatrix(genl2),1)

%     for i in range(m-1):
%         for j in range(i+1, m):
%             [r,c] = np.where(M2 == M1[i,j])
%             for k in range(len(r)):
%                 VT[(i)*n + r[k]] = 1;
%                 VT[(i)*n + c[k]] = 1;
%                 VT[(j)*n + r[k]] = 1;
%                 VT[(j)*n + c[k]] = 1;

%                 if M is None:
%                     M = np.copy(VT)
%                 else:
%                     M = np.concatenate((M, VT), 1)

%                 VT = np.zeros((n*m,1), int)

%     return M
% \end{lstlisting}
