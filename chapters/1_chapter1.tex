%%
% The BIThesis Template for Bachelor Graduation Thesis
%
% 北京理工大学毕业设计(论文)第一章节 —— 使用 XeLaTeX 编译
%
% Copyright 2020-2021 BITNP
%
% This work may be distributed and/or modified under the
% conditions of the LaTeX Project Public License, either version 1.3
% of this license or (at your option) any later version.
% The latest version of this license is in
%   http://www.latex-project.org/lppl.txt
% and version 1.3 or later is part of all distributions of LaTeX
% version 2005/12/01 or later.
%
% This work has the LPPL maintenance status `maintained'.
%
% The Current Maintainer of this work is Feng Kaiyu.
%
% 第一章节

\chapter{绪论}

\section{研究背景与意义}
% 这里插入一个参考文献,仅作参考
% 正文……\cite{yuFeiJiZongTiDuoXueKeSheJiYouHuaDeXianZhuangYuFaZhanFangXiang2008}
自动驾驶系统的决策模块需要先进的决策算法保证安全性、智能性、有效性。目前传统算法的解决思路是以价格昂贵的激光雷达作为主要传感器,依靠人工设计的算法从复杂环境中提取关键信息, 根据这些信息进行决策和判断。该算法缺乏一定的泛化能力,不具备应有的智能性和通用性。深度强化学习的出现有效地改善了传统算法泛化性不足的问题, 这给智能驾驶领域带来新的思路。

目前,强化学习在策略选择的理论和算法方面已经取得了很大的进步,然而直接从高维感知输入(如图像、语音等)中提取特征,学习最优策略,对强化学习来说依然是一个挑战。

深度强化学习(Deep Reinforcement Learning, DRL)结合了深度神经网络和强化学习的优势, 可以用于解决智能体在复杂高维状态空间中的感知决策问题\cite{唐振韬2017深度强化学习进展}\cite{2017Deep}。2016年,基于深度强化学习和蒙特卡洛树搜索的AlphaGo击败了人类顶尖职业棋手,引起了全世界的关注\cite{2017Review}。 2017年, DeepMind在《Nature》上公布了最新版AlphaGo论文, 介绍了更强的围棋人工智能: AlphaGo Zero。它不需要人类专家知识, 只使用纯粹的深度强化学习技术和蒙特卡罗树搜索, 经过3天自我对弈就以100比0击败了上一版本的AlphaGo。AlphaGo Zero证明了深度强化学习的强大能力, 也必将推动以深度强化学习为代表的人工智能领域的进一步发展。基于深度强化学习在棋局与游戏上的成功,最近的研究大多注重于深度强化学习在各个领域中的扩展与应用。



\section{国内外研究现状}

\section{本文主要研究内容及章节安排}

% \subsection{三级题目}

% 正文……\cite{Hajela2012Application}

% \begin{figure}[htbp]
%   \vspace{13pt} % 调整图片与上文的垂直距离
%   \centering
%   \includegraphics[]{images/bit_logo.png}
%   \caption{标题序号}\label{标题序号} % label 用来在文中索引
% \end{figure}

% \textcolor{blue}{\underline{\underline{表-示例:(阅后删除此段)}}}

% \begin{table}[htbp]
%   \linespread{1.5}
%   \zihao{5}
%   \centering
%   \caption{统计表}\label{统计表}
%   \begin{tabular}{*{5}{>{\centering\arraybackslash}p{2cm}}}
%     \hline
%     项目    & 产量    & 销量    & 产值   & 比重    \\ \hline
%     手机    & 1000  & 10000 & 500  & 50\%  \\
%     计算机   & 5500  & 5000  & 220  & 22\%  \\
%     笔记本电脑 & 1100  & 1000  & 280  & 28\%  \\ \hline
%     合计    & 17600 & 16000 & 1000 & 100\% \\ \hline
%     \end{tabular}
% \end{table}

% \textcolor{blue}{公式标注应于该公式所在行的最右侧。对于较长的公式只可在符号处(+、-、*、/、$\leqslant$ $\geqslant$ 等)转行。在文中引用公式时,在标号前加“式”,如式(1-2)。阅后删除此
% 段。}

% \textcolor{blue}{公式-示例:(阅后删除此段)}
% % 公式上下不要空行,置于同一个段落下即可,否则上下距离会出现高度不一致的问题
% \begin{equation}
%     LRI=1\ ∕\ \sqrt{1+{\left(\frac{{\mu }_{R}}{{\mu }_{s}}\right)}^{2}{\left(\frac{{\delta }_{R}}{{\delta }_{s}}\right)}^{2}}
% \end{equation}

% \subsubsection{生僻字}

% % 一个可能无法正常显示的生僻字
% 一个可能无法正常显示的生僻字: 彧。下文注释中,介绍了如何通过自定义字体来显示生僻字。

% % 定义一个提供了生僻字的字体,注意要确保你的系统存在该字体
% % \setCJKfamilyfont{custom-font}{Noto Serif CJK SC}

% % 使用自己定义的字体
% % 使用提供了相应字型的字体:\CJKfamily{custom-font}{彧}。

